\documentclass[12pt,paper=a4]{scrartcl}

%%%%%%%%%%%%%%%%%%%%%%%%%%%%%%%%%%%%%%%%%%%%%%%%%%%%%%%%
% Deutsche Silbentrennung, Datumsangaben, Schrift, ... %
%%%%%%%%%%%%%%%%%%%%%%%%%%%%%%%%%%%%%%%%%%%%%%%%%%%%%%%%
\usepackage[ngerman]{babel}

% Umlaute korrekt interpretieren und anzeigen 
\usepackage[utf8]{inputenc}
\usepackage[T1]{fontenc}
\usepackage{lmodern}

% Anführungsstriche
\usepackage{csquotes}
\MakeOuterQuote{"}

% Leerzeile und keine Einrückung zwischen Absätzen
\usepackage[parfill]{parskip}



%%%%%%%%%%%%%%%%
% Seitenlayout %
%%%%%%%%%%%%%%%%
\usepackage[a4paper,left=25mm,right=25mm,top=25mm,bottom=20mm]{geometry}
\usepackage{nopageno}



%%%%%%%%%%%%
% Grafiken %
%%%%%%%%%%%%
\usepackage{graphicx} 
\usepackage{float}



%%%%%%%%%%%%%%%%%
% Verschiedenes %
%%%%%%%%%%%%%%%%%
% URLs mit \url{http://google.de/} verwenden
\usepackage{url}

% PDF Infos
\usepackage{hyperref}
\hypersetup{
	breaklinks=true,
	colorlinks=true,
	pdfstartview=Fit,
	pdfpagelayout=SinglePage,
	filecolor=darkblue,
	urlcolor=darkblue,
	linkcolor=black,
	citecolor=black
}

% Mit \cref{} \label{}-Definitionen referenzieren
\usepackage[ngerman,capitalise,nameinlink]{cleveref}



%%%%%%%%%%%%
% Commands %
%%%%%%%%%%%%
\makeatletter
\newcommand{\seminar}[1]{\renewcommand{\@seminar}{#1}}
\newcommand{\@seminar}[1]{}
\newcommand{\keywords}[1]{\renewcommand{\@keywords}{#1}}
\newcommand{\@keywords}[1]{}
\makeatother


\begin{document}
	
\seminar{
	Hauptseminar CAD/CAM-Technologie:\\
	"Anforderungsspezifikationen in der Entwicklung technischer Produkte"
}
\title{Titel des Beitrags}
\author{Name des Autors}
\keywords{3--5 Schlüsselwörter in kursiv}

\makeatletter

\hypersetup{
	pdftitle={\@title}, 
	pdfauthor={\@author},
	pdfkeywords={\@keywords}
}

\begin{center}
	\fontsize{14pt}{16pt}\selectfont
	\@seminar

	\vspace{24pt}

	\MakeUppercase{\@title}

	\vspace{24pt}

	\normalsize
	\@author
\end{center}

\vspace{24pt}
\textit{Schlüsselwörter: \@keywords}

\makeatother



%\begin{abstract}
%	\textbf{Wie soll die Kurzfassung aussehen??}
%	Ut wisi enim ad minim veniam, quis nostrud exerci tation ullamcorper suscipit lobortis nisl ut aliquip ex ea commodo consequat. Duis autem vel eum iriure dolor in hendrerit in vulputate velit esse molestie consequat, vel illum dolore eu feugiat nulla facilisis at vero eros et accumsan et iusto odio dignissim qui blandit praesent luptatum zzril delenit augue duis dolore te feugait nulla facilisi.
%\end{abstract}




\section{Einleitung}
Die folgenden Anweisungen legen dar, in welcher Form Sie Ihr Manuskript zur Abgabe vorzubereiten haben.
Als Textverarbeitungssystem sollte nach Möglichkeit Microsoft Word oder StarOffice verwendet werden.

Der Beitrag muss zum Zeitpunkt in druckreifer Form vorliegen.
Vervielfältigungen werden direkt von Ihrer Vorlage angefertigt.
Daher müssen sich alle Illustrationen, Grafiken, Diagramme, und Tabellen etc.\ an der richtigen Stelle im Text befinden.
An dem abzugebenden Manuskript können keine Änderungen oder Korrekturen vorgenommen werden.
Damit Ihr Beitrag im Tagungsband möglichst gut aussieht, sind die im Folgenden ausgeführten Anweisungen strikt zu befolgen.
Die fertige Ausarbeitung senden Sie bitte per Email als Dokument (DOC oder ODT) an: mail.des.betreuers@informatik.uni-stuttgart.de



\section{Zu befolgende Formatierungsvorgaben}
\subsection{Allgemeine Bemerkungen}
\begin{itemize}
	\item Das Manuskript darf keine Seitennummern enthalten!
	\item Es dürfen weder Kopf- noch Fußzeilen verwendet werden!
	\item Der Beitrag sollte ungefähr zwischen 15--20 Seiten umfassen.
	\item Das erste Kapitel nach den Schlüsselwörtern sollte "Einleitung" heißen. Das letzte Kapitel sollte "Schlussfolgerung und Ausblick" heißen.
	\item Die Länge der Kapitel "Einleitung" sowie "Schlussfolgerung und Ausblick" sollte maximal 10--14 Zeilen umfassen.
	\item Sie können der "Einleitung" eine Kurzfassung des Beitrags voranstellen.
	\item Um eine gute Lesbarkeit der Ausarbeitung zu garantieren, dürfen keine schattierten Texte verwendet werden.
\end{itemize}


\subsection{Seiten-Layout}
\begin{itemize}
	\item Das Layout des Kopfteils der Ausarbeitung hat in oben dargestellter Weise zu erfolgen.
	\item Blattformat: DIN A4
	\item Blocksatz, Silbentrennung
	\item Seitenränder (mm): 25 links, 25 rechts, 25 oben, 20 unten
\end{itemize}


\subsection{Schrifttyp, der Schriftgröße und Abstände}
...
\pagebreak


\subsection{Grafiken und Tabellen}
Alle Grafiken, Fotos etc. sollten in der Mitte der Seite (zentriert) erscheinen.
Sie dürfen keinesfalls größer als der Textbereich sein.
Die Legenden sollten wie unten angezeigt nummeriert werden.
Bitte beachten Sie, dass der Seminar-Gesamtbericht in schwarz/weiß gedruckt werden wird.
Optimieren sie Ihre Grafiken in dieser Hinsicht.

\begin{figure}[h]
	\centering
	\includegraphics[height=3.5cm]{graphics/uni-stuttgart-logo}
	\caption{Das Logo der Universität Stuttgart}
	\label{fig:uni-stuttgart-logo}
\end{figure}

Tabellen sollten eine Tabellenüberschrift vorausgehen.

\begin{table}[h]
	\centering
	\caption{Beispiel einer Tabelle}
	\begin{tabular}{|p{5cm}|p{5cm}|} \hline
		 & \\ \hline
		 & \\ \hline
	\end{tabular}
\end{table}

\subsection{Aufzählungen}
Beispiel einer Aufzählung mit Punkten:
\begin{itemize}
	\item Aufzählungseintrag mit Punkt
\end{itemize}
Beispiel einer Aufzählung mit Spiegelstrichen:
\begin{itemize}
	\item[--] Aufzählungseintrag mit Spiegelstrich
\end{itemize}
Beispiel einer Aufzählung mit Nummerierung:
\begin{enumerate}
	\item Aufzählungseintrag mit Nummerierung
\end{enumerate}


\subsection{Gleichungen}
Beispiel einer Gleichung mit Nummerierung (rechtsbündig) in runden Klammern:
\begin{equation}
	Dies \frac{ist}{eine}Gleichung
\end{equation}


\pagebreak
\section{Abschlussbemerkungen}\label{sec:abschluss}
Ein Beispiel für den Schluss der Ausarbeitung einschließlich der Literaturangaben, der Autoren, sowie dessen Adresse wird nach diesen Abschlussbemerkungen angezeigt.
Die Überschrift der Literaturangaben wird nicht nummeriert.
Nummerierungen zu Literaturangaben sollten in eckige Klammern eingefasst werden.
Im Text sollten Literaturstellen in gleicher Weise zitiert werden.
Bitte formatieren Sie verschiedenen Literaturarten wie folgt:

...

Müller et al.\ zeigen in \cite{article}, dass der Ansatz korrekt ist \cite{inproceeding, inproceeding2, inproceeding3}.

\begin{thebibliography}{}
	\bibitem{article}
	Beispiel einer Literaturangabe.
	\bibitem{inproceeding}
	Nachname, X, "Titel des Artikels...", Proceedings of the 11th International Conference on Engineering Design in Tampere, Volume 1, Schriftenreihe WDK 25, Tampere 1997, S.\ 756--761
	\bibitem{inproceeding3}
	Nachname, X, "Titel 3", Proceedings of the 11th International Conference on Engineering Design in Tampere, Volume 1, Schriftenreihe WDK 25, Tampere 1997, S.\ 756--761
	\bibitem{inproceeding2}
	Nachname, X, "Titel 2", Proceedings of the 11th International Conference on Engineering Design in Tampere, Volume 1, Schriftenreihe WDK 25, Tampere 1997, S.\ 756--761
\end{thebibliography}




\pagebreak

\section{Weiteres}
Referenzen innerhalb des Dokuments mit \texttt{\textbackslash cref}, sodass für die \textit{Abschlussbemerkungen} sowas entsteht: \cref{sec:abschluss}.

URLs gehen auch: \url{http://www.google.de/}.
\end{document}
