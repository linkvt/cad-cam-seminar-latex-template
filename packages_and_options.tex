%%%%%%%%%%%%%%%%%%%%%%%%%%%%%%%%%%%%%%%%%%%%%%%%%%%%%%%%
% Deutsche Silbentrennung, Datumsangaben, Schrift, ... %
%%%%%%%%%%%%%%%%%%%%%%%%%%%%%%%%%%%%%%%%%%%%%%%%%%%%%%%%
\usepackage[ngerman]{babel}

% Umlaute korrekt interpretieren und anzeigen 
\usepackage[utf8]{inputenc}
\usepackage[T1]{fontenc}
\usepackage{lmodern}

% Anführungsstriche
\usepackage{csquotes}
\MakeOuterQuote{"}

% Leerzeile und keine Einrückung zwischen Absätzen
\usepackage[parfill]{parskip}



%%%%%%%%%%%%%%%%
% Seitenlayout %
%%%%%%%%%%%%%%%%
\usepackage[a4paper,left=25mm,right=25mm,top=25mm,bottom=20mm]{geometry}
\usepackage{nopageno}



%%%%%%%%%%%%
% Grafiken %
%%%%%%%%%%%%
\usepackage{graphicx} 
\usepackage{float}



%%%%%%%%%%%%%%%%%
% Verschiedenes %
%%%%%%%%%%%%%%%%%
% URLs mit \url{http://google.de/} verwenden
\usepackage{url}

% PDF Infos
\usepackage{hyperref}
\hypersetup{
	breaklinks=true,
	colorlinks=true,
	pdfstartview=Fit,
	pdfpagelayout=SinglePage,
	filecolor=darkblue,
	urlcolor=darkblue,
	linkcolor=black,
	citecolor=black
}

% Mit \cref{} \label{}-Definitionen referenzieren
\usepackage[ngerman,capitalise,nameinlink]{cleveref}



%%%%%%%%%%%%
% Commands %
%%%%%%%%%%%%
\makeatletter
\newcommand{\seminar}[1]{\renewcommand{\@seminar}{#1}}
\newcommand{\@seminar}[1]{}
\newcommand{\keywords}[1]{\renewcommand{\@keywords}{#1}}
\newcommand{\@keywords}[1]{}
\makeatother
